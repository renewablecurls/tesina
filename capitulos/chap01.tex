\chapter{Título de capítulo de introducción}

El presente documento establece los aspectos más importantes del formato de las tesinas a realizar en el marco de la Asignatura \textit{Proyecto de Investigación e Innovación en Ingeniería Estructural} de la carrera Ingeniería Civil de la Facultad de Ingeniería de la Universidad de la República. %

Este documento fue generado utilizando una clase de \LaTeX \, llamada \textit{tesinapiiie.cls} la cual fue generada a través de modificaciones en la clase \textit{udelartex.cls}. \footnote{La clase UdelaRTeX fue desarrollada por la Comisión Académica de Posgrado de la Universidad de la República para la realización de tesis de posgrado e implementada por los docentes del IET P. Castrillo y M. Caballero, por más información visitar \url{http://tesis.posgrados.udelar.edu.uy/TallerTesis/UdelaRTeX}.}

El uso de la clase \textit{tesinapiiie} permite a los estudiantes enfocarse en la generación de contenido ya que todo el estilo de la tesina es automáticamente generado por \LaTeX \, así como también se recomienda que la bibliografía sea introducida a través de archivos .bib generados con gestores de bibliografía adecuados. %



\section{Formato del texto, esto es un título de sección}

En el caso de preferir usar otro procesador de textos los estudiantes deberán aplicar el formato definido en este documento. %
%
A continuación se destacan algunos puntos importantes del estilo.

La fuente utilizada para el texto normal es:
\begin{itemize}
	\item \textit{mathptmx} en \LaTeX 
	\item \textit{Times new roman} en MS-Office
	\item \textit{Liberation serif} en LibreOffice
\end{itemize}
%
o equivalentes en otros editores de texto. %
El tamaño de la fuente de texto normal es 12. %
%
El interlineado debe tener un factor de $1.5$. %

En la primer página, el título va en fuente normal tamaño 21 y los logos de las instituciones tienen una altura de 16 mm.

Las versiones de la tesina que vayan a ser revisadas o evaluadas deben tener todas las lineas numeradas como se muestra en este documento. %
%
Estas lineas no deberán estar en la versión final a entregar en biblioteca.

\subsection{Secciones, esto es un título de subsección}


Los títulos de capítulos en negrita y tamaño 24. %
%
Los títulos de las secciones en negrita y tamaño 17.
%
Los títulos de las subsecciones en negrita y tamaño 14.

Los títulos o descripción de las figuras y tablas se escriben en tamaño 11 y usando la negrita como se ve en el ejemplo de la \autoref{tab:comp} .

\subsubsection{Subsubsección}

Los títulos de subsubsecciones van en tamaño 12 y negrita.

\section{Hoja}

La hoja a usar es A4 y los márgenes son 

\begin{itemize}
  \item superior: 3.5cm
  \item inferior: 3cm
  \item izquierdo: 4.0cm
  \item derecho: 3.0cm
\end{itemize}


\section{Estilo de citación y referencias}

Para las referencias bibliográficas se utiliza el estilo definido por la \textit{American Psychology Association} APA \footnote{Ejemplos de la norma se pueden ver en \url{https://owl.english.purdue.edu/media/pdf/20110928111055_949.pdf}}.

Para mostrar ejemplos en la tesina se pueden citar por ejemplo artículos \citep{article-example}, o libros \citep{book-example}, usando el comando \verb|\citep| de \LaTeX \, o también citar al autor utilizando el comando \verb|\cite|, como: \cite{article-example}.

También se pueden citar varios documentos: \citep{Rosati2008,Krour2013,LeMagorou2002}.


